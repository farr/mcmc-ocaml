\documentclass{article}

\begin{document}

The Bayesian evidence, $p(d|M)$, for a model with parameters $\theta$,
likelihood $p(d|\theta,M)$, prior $p(\theta|M)$ given data $d$ is
\begin{equation}
  p(d|M) = \int d\theta\, p(d|\theta,M) p(\theta|M).
\end{equation}
Thermodynamic integration is an attempt to compute the evidence using
a sequence of ``tempered'' likelihoods: 
\begin{equation}
  p(d|\theta,\beta,M) \equiv p(d|\theta,M)^\beta,
\end{equation}
with $0 \leq \beta \leq 1$.  (The parameter $\beta$ plays the role of
an inverse temperature: $\beta = 1/T$ for $1 \leq T < \infty$; high
temperatures---small $\beta$s---result in smoothed likelihoods, which
can help a sequence of samples from the posterior avoid getting stuck
in a sharp (local) maximum.)  Define a partition function, $Z(\beta)$,
which is the evidence evaluated with the tempered likelihood:
\begin{equation}
  Z(\beta) \equiv \int d\theta\, p(d|\theta,\beta,M) p(\theta|M) =
  \int d\theta\, p(d|\theta,M)^\beta p(\theta|M).
\end{equation}
(Note that the prior is not tempered in $Z(\beta)$.)

The partition function satisfies a differential equation 
\begin{equation}
  \frac{dZ}{d\beta} = \int d\theta \log(p(d|\theta,M))
  p(d|\theta,\beta,M) p(\theta|M),
\end{equation}
or, upon dividing both sides by $Z(\beta)$,
\begin{equation}
  \frac{d\log Z}{d\beta} = \langle \log(p(d|\theta,M)) \rangle_\beta,
\end{equation}
where $\langle \ldots \rangle_\beta$ is the expectation value over the
tempered posterior.  Using the boundary condition
\begin{equation}
  Z(0) = 1,
\end{equation}
we can integrate this equation to obtain 
\begin{equation}
  Z(1) = p(d|M) = \int_0^1 d\beta\, \langle p(d|\theta,M) \rangle_\beta.
\end{equation}

If we run a parallel-tempered MCMC, we have some number of chains that
sample the tempered posterior at various values of $\beta$.  Using
these samples, we can approximate the evidence integral using some
quadrature scheme.  

\end{document}